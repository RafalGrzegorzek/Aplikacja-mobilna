\newpage
\section{Określenie wymagań szczegółowych}		%2
%Dokładne określenie wymagań aplikacji (cel, zakres, dane wejściowe) – np. opisać przyciski, czujniki, wygląd layautu, wyświetlenie okienek. Opisać zachowanie aplikacji – co po kliknięciu, zdarzenia automatyczne. Opisać możliwość dalszego rozwoju oprogramowania. Opisać zachowania aplikacji w niepożądanych sytuacjach.
\begin{itemize}
	\item Analiza wymagań: Przeprowadzenie szczegółowej analizy wymagań klienta i przetworzenie ich na konkretny plan projektu.
	\item Projektowanie interfejsu użytkownika: Zaplanowanie interfejsu użytkownika, uwzględniając ergonomię, intuicyjność i estetykę, aby zapewnić jak najlepsze doświadczenie dla użytkownika.
	\item Konfiguracja środowiska : Utworzenie projektu Android Studio w odpowiednim IDE, takim jak Visual Studio, skonfigurowanie narzędzi i środowiska pracy.
	\begin{figure}[!hbt]
		\begin{center}
			\includegraphics[width=5cm]{rys/android-studio.png}
			\caption{Android Studio}
			\label{rys:Android-studio}
		\end{center}
	\end{figure}
	\item Integracja z aparatem: Wykorzystanie funkcji aparatu do skanowania kodów QR produktów, oraz przekierowywanie do stron WWW.
	\item Integracja z usługami map: Wykorzystanie usług map, takich jak Google Maps API lub Mapy Apple, w celu wyświetlania lokalizacji sklepów spożywczych oraz planowania tras do nich.
	\item Testowanie: Regularne testowanie aplikacji w celu zapewnienia, że wszystkie funkcje działają zgodnie z oczekiwaniami i nie ma błędów.
	\item Optymalizacja: Ulepszanie wydajności i responsywności aplikacji poprzez optymalizację kodu i zasobów, aby zapewnić płynne działanie.
\end{itemize}



