	\newpage
\section{Implementacja}		%4
%Wkleić szkielet kodu, wraz z komentarzami. Opisać zmienne, struktury do czego służą. Opisać procedury, metody co wykonują. Opisać nowe zdefiniowane klasy. Opisać dziedziczenie. Opisać nowo utworzone pliki za co odpowiadają.
\subsection{Pliki layout'u}
\hspace{0.60cm}Layout w naszej aplikacji podzielony został na 4 pliki.xml:
\begin{itemize}
	\item activity\_start.xml
	\item activity\_main.xml
	\item fragment\_about.xml
        \item fragment\_name\_input.xml
\end{itemize}

\subsubsection{activity\_start}
\begin{lstlisting}[caption=Activity\_start, label={lst:kod.xml}, language=XML]
	<?xml version="1.0" encoding="utf-8"?>
	<androidx.constraintlayout.widget.ConstraintLayout xmlns:android="http://schemas.android.com/apk/res/android"
	xmlns:app="http://schemas.android.com/apk/res-auto"
	xmlns:tools="http://schemas.android.com/tools"
	android:layout_width="match_parent"
	android:layout_height="match_parent"
	android:background="@drawable/background3">
	
	
	<Button
	android:id="@+id/buttonstart"
	android:layout_width="297dp" 
	android:layout_height="101dp" 
	android:layout_marginStart="57dp"
	android:layout_marginTop="334dp"
	android:layout_marginEnd="57dp"
	android:layout_marginBottom="296dp"
	android:backgroundTint="#0C390F"
	android:fontFamily="serif-monospace"
	android:text="Start"
	android:textSize="48sp"
	app:layout_constraintBottom_toBottomOf="parent"
	app:layout_constraintEnd_toEndOf="parent"
	app:layout_constraintStart_toStartOf="parent"
	app:layout_constraintTop_toTopOf="parent" />
	
	</androidx.constraintlayout.widget.ConstraintLayout>
\end{lstlisting}
\newpage
\hspace{0.60cm}W listungu nr. 1:
\begin{itemize}
	\item Główny kontener, który używa ConstraintLayout (część biblioteki AndroidX) do definiowania układu komponentów interfejsu. Jest to rodzaj layoutu, który pozwala na definiowanie relacji pomiędzy komponentami - przedstawiony w liniach od 2 do 7.
	\item W liniach od 10 do 25 opisany został przycisk rozpoczynający activity\_main. Jego właściwości:
	\begin{itemize}
		\item android:layout\_width i android:layout\_height: Szerokość i wysokość przycisku.
		\item android:layout\_marginStart, android:layout\_marginTop, android:layout\_marginEnd, android:layout\_marginBottom: Marginesy przycisku od krawędzi layoutu.
		\item android:backgroundTint= \#0C390F: Kolor tła przycisku.
		\item android:fontFamily="serif-monospace": Wybrana rodzina czcionek.
		\item android:text="Start": Tekst wyświetlany na przycisku.
		\item android:textSize="48sp": Rozmiar tekstu.
		\item app:layout\_constraint*: Definicje ograniczeń umieszczające przycisk na ekranie. Przycisk jest przytwierdzony do krawędzi layoutu, co oznacza, że będzie zajmować całą dostępną przestrzeń.
	\end{itemize}
\end{itemize}
\subsubsection{activity\_main}
\begin{lstlisting}[caption=Activity\_main, label={lst:kod.xml}, language=XML]
	<?xml version="1.0" encoding="utf-8"?>
	<androidx.constraintlayout.widget.ConstraintLayout
	xmlns:android="http://schemas.android.com/apk/res/android"
	xmlns:app="http://schemas.android.com/apk/res-auto"
	xmlns:tools="http://schemas.android.com/tools"
	android:layout_width="match_parent"
	android:layout_height="match_parent"
	android:background="@drawable/background1"
	tools:context=".MainActivity">
	
	<FrameLayout
	android:id="@+id/fragment_container"
	android:layout_width="match_parent"
	android:layout_height="0dp"
	app:layout_constraintTop_toTopOf="parent"
	app:layout_constraintBottom_toBottomOf="parent"
	app:layout_constraintStart_toStartOf="parent"
	app:layout_constraintEnd_toEndOf="parent"/>
	
	
	<Button
	android:id="@+id/btnCam"
	android:layout_width="250dp"
	android:layout_height="80dp"
	android:layout_marginStart="50dp"
	android:layout_marginTop="380dp"
	android:layout_marginEnd="50dp"
	android:layout_marginBottom="40dp"
	android:backgroundTint="#062507"
	android:fontFamily="monospace"
	android:text="Skanuj QR"
	android:textAppearance="@style/TextAppearance.AppCompat.Large"
	app:layout_constraintBottom_toTopOf="@+id/button4"
	app:layout_constraintEnd_toEndOf="parent"
	app:layout_constraintHorizontal_bias="0.488"
	app:layout_constraintStart_toStartOf="parent"
	app:layout_constraintTop_toTopOf="parent" />
	
	<Button
	android:id="@+id/button2"
	android:layout_width="250dp"
	android:layout_height="80dp"
	android:layout_marginStart="50dp"
	android:layout_marginTop="260dp"
	android:layout_marginEnd="50dp"
	android:layout_marginBottom="40dp"
	android:backgroundTint="#062507"
	android:fontFamily="monospace"
	android:text="Szukaj"
	android:textAppearance="@style/TextAppearance.AppCompat.Large"
	app:layout_constraintBottom_toTopOf="@+id/btnCam"
	app:layout_constraintEnd_toEndOf="parent"
	app:layout_constraintHorizontal_bias="0.488"
	app:layout_constraintStart_toStartOf="parent"
	app:layout_constraintTop_toTopOf="parent" />
	
	<ImageView
	android:id="@+id/imageView"
	android:layout_width="match_parent"
	android:layout_height="match_parent"
	android:scaleType="centerCrop"
	android:visibility="invisible"
	tools:layout_editor_absoluteX="0dp"
	tools:layout_editor_absoluteY="0dp" />
	
	<TextView
	android:id="@+id/textView"
	android:layout_width="wrap_content"
	android:layout_height="wrap_content"
	android:layout_marginStart="16dp"
	android:layout_marginTop="16dp"
	app:layout_constraintStart_toStartOf="parent"
	app:layout_constraintTop_toBottomOf="@+id/button4" />
	
	<ImageView
	android:id="@+id/imageView3"
	android:layout_width="393dp"
	android:layout_height="131dp"
	android:layout_marginBottom="129dp"
	android:scaleType="centerCrop"
	app:layout_constraintBottom_toTopOf="@+id/button2"
	app:layout_constraintEnd_toEndOf="parent"
	app:layout_constraintStart_toStartOf="parent"
	app:layout_constraintTop_toTopOf="parent"
	app:srcCompat="@mipmap/ic_launcher4_foreground" />
	
	<Button
	android:id="@+id/button4"
	android:layout_width="250dp"
	android:layout_height="80dp"
	android:layout_marginStart="50dp"
	android:layout_marginTop="500dp"
	android:layout_marginEnd="50dp"
	android:backgroundTint="#062507"
	android:fontFamily="monospace"
	android:text="Informacje"
	android:textAppearance="@style/TextAppearance.AppCompat.Large"
	app:layout_constraintEnd_toEndOf="parent"
	app:layout_constraintHorizontal_bias="0.488"
	app:layout_constraintStart_toStartOf="parent"
	app:layout_constraintTop_toTopOf="parent" />
	
	
	</androidx.constraintlayout.widget.ConstraintLayout>
\end{lstlisting}
\hspace{0.60cm} W projekcie, wygląd strony głównej jest określony w pliku "activity\_main", w listingu 2 opisane są jej właściwości.
\begin{itemize}
	\item W liniach od 2 do 9 opisany jest korzeń layoutu:
	\begin{itemize}
		\item ConstraintLayout: Główny layout, który pozwala na definiowanie relacji pomiędzy komponentami.
		\item xmlns:android, xmlns:app, xmlns:tools: Przestrzenie nazw XML używane w dokumencie.
		\item android:layout\_width i android:layout\_height: Szerokość i wysokość layoutu zajmują całą dostępną przestrzeń.
		\item android:background: Ustawia tło layoutu na obrazek zdefiniowany w pliku background1.
		\item tools:context=".MainActivity": Dodatkowe narzędzia dla środowiska deweloperskiego, takie jak podpowiedzi dotyczące kontekstu.
	\end{itemize}
	\item Linie 11 do 18, kontener, który jest używany do wyświtlania fragmentóW.
	\item Przycisk, który pozwala na skanowanie Kodów QR jest opisany w liniach od 21 do 37.
	\item W liniach od 39 do 55 znajduje się przycisk "Szukaj", który jest główną funkcjonalnością aplikacji. Pozwala on na wyszukiwanie za pomocą GPS-u pobliskich sklepów "Żabka".
	\item TextView, w liniach 66 do 73 użyty został do wyświetlania tekstu.
	\item W liniach od 75 do 85 opisany jest obrazek, wyświetlany w main activity.
	\item Od 87 do 101 lini określony został przycisk "Informacje", który przekierowuje nas do kolejnej strony.
\end{itemize}
\subsubsection{fragment\_about}
\begin{lstlisting}[caption=Fragment\_about, label={lst:kod.xml}, language=XML]
	<LinearLayout xmlns:android="http://schemas.android.com/apk/res/android"
	android:layout_width="match_parent" android:layout_height="match_parent"
	android:background="@drawable/background2" android:gravity="center"
	android:orientation="vertical" android:padding="16dp">
	<Button android:id="@+id/button11" android:layout_width="250dp"
	android:layout_height="80dp"
	android:backgroundTint="#283593"
	android:fontFamily="sans-serif-light"
	android:text="O nas"
	android:textAppearance="@style/TextAppearance.AppCompat.Large"/>
	
	<Button android:id="@+id/button12"
	android:layout_width="250dp"
	android:layout_height="80dp"
	android:layout_marginTop="20dp"
	android:backgroundTint="#283593"
	android:fontFamily="sans-serif-light"
	android:text="Wersja"
	android:textAppearance="@style/TextAppearance.AppCompat.Large"/>
	
	<Button android:id="@+id/button13"
	android:layout_width="250dp"
	android:layout_height="80dp"
	android:layout_marginTop="20dp"
	android:backgroundTint="#283593"
	android:fontFamily="sans-serif-light"
	android:text="Pomoc"
	android:textAppearance="@style/TextAppearance.AppCompat.Large"/>
	
	<Button android:id="@+id/button14"
	android:layout_width="250dp"
	android:layout_height="80dp"
	android:layout_marginTop="20dp"
	android:backgroundTint="#283593"
	android:fontFamily="sans-serif-light"
	android:text="Powrot"
	android:textAppearance="@style/TextAppearance.AppCompat.Large"/>
	</LinearLayout>
\end{lstlisting}
\hspace{0.60cm} Podstrona wyświetlana po naciśnięciu przycisku Informacje zawiera następujące definicje:

\begin{itemize}
	\item Główny układ (LinearLayout):
	\begin{itemize}
		\item xmlns:android="http://schemas.android.com/apk/res/android": Ta \\
		deklaracja przestrzeni nazw XML jest wymagana w plikach XML dla Androida.
		\item android:layout\_width="match\_parent" i android:layout\_height="match\_parent": Układ zajmuje całą dostępną szerokość i wysokość swojego rodzica.
		\item android:background="@drawable/background2": Ustawia tło układu na zasób graficzny o nazwie "background2".
		\item android:gravity="center": Wyrównuje dzieci układu do środka.
		\item android:orientation="vertical": Układ rozmieszcza swoje dzieci w układzie pionowym.
		\item android:padding="16dp": Dodaje 16 pikseli paddingu niezależnego od gęstości wokół układu.
	\end{itemize}
	\item Przyciski:
	\begin{itemize}
		\item Cztery elementy Button reprezentują przyciski do kliknięcia w aplikacji.
		\item Każdy przycisk ma unikalne ID (@+id/button11, @+id/button12, \\@+id/button13, @+id/button14).
	\end{itemize}
	\item Właściwości każdego przycisku:
	\begin{itemize}
		\item android:layout\_width="250dp" i android:layout\_height="80dp": Ustawiają szerokość i wysokość przycisku.
		\item android:backgroundTint="\#283593": Zmienia kolor tła przycisku na odcień niebieskiego.
		\item android:fontFamily="sans-serif-light": Określa rodzinę czcionek dla tekstu przycisku.
		\item android:text: Ustawia tekst na przycisku ("O nas", "Wersja", "Pomoc", "Powrót").
		item android:textAppearance="\\@style/TextAppearance.AppCompat.Large": Określa wygląd tekstu, w tym jego rozmiar, dla dużego ekranu.
	\end{itemize}
\end{itemize}

\subsubsection{fragment\_name\_input.xml}
\begin{lstlisting}
[caption=Fragment\_name\_input.xml, label={lst:kod.xml}, language=XML]
<LinearLayout
    xmlns:android="http://schemas.android.com/apk/res/android"
    android:background="@drawable/background1"
    android:layout_width="match_parent"
    android:layout_height="match_parent"
    android:orientation="vertical"
    android:padding="16dp">

    <EditText
        android:id="@+id/etName"
        android:layout_width="match_parent"
        android:layout_height="wrap_content"
        android:hint="Wprowadz imie"
        android:padding="20dp" />

    <Button
        android:id="@+id/btnSave"
        android:layout_width="wrap_content"
        android:layout_height="wrap_content"
        android:text="Zapisz" />

    <Button
        android:id="@+id/buttonback"
        android:layout_width="100dp"
        android:layout_height="50dp"
        android:layout_marginTop="500dp"
        android:backgroundTint="#283593"
        android:fontFamily="sans-serif-light"
        android:text="@string/button14"
        android:textAppearance="@style/TextAppearance.AppCompat.Small" />
</LinearLayout>
\end{lstlisting}
\begin{itemize}
    \item LinearLayout:
    \begin{itemize}
        \item xmlns:android: Ustala przestrzeń nazw dla atrybutów Android.
        \item android:background: Ustawia tło interfejsu na obraz zasobu o nazwie "background1".
        \item android:layout\_width i android:layout\_height: Określają szerokość i wysokość layoutu na "match\_parent", co oznacza, że mają zajmować całą dostępną przestrzeń.
        \item android:orientation: Określa orientację dzieci w LinearLayout jako pionową.
        \item android:padding: Ustawia wewnętrzny odstęp dla całego layoutu na 16dp.
    \end{itemize}
    \item EditText:
    \begin{itemize}
        \item android:id: Unikalny identyfikator dla pola tekstowego.
        \item android:layout\_width i android:layout\_height: Określają szerokość i wysokość pola tekstowego na "match\_parent" i "wrap\_content" odpowiednio.
        \item android:hint: Tekst podpowiedzi wyświetlany w polu tekstowym.
        \item android:padding: Ustawia wewnętrzny odstęp dla pola tekstowego na 20dp.
    \end{itemize}
    \item Button (btnSave):
    \begin{itemize}
        \item android:id: Unikalny identyfikator dla przycisku "Zapisz".
        \item android:layout\_width i android:layout\_height: Określają szerokość i wysokość przycisku na "wrap\_content".
        \item android:text: Tekst wyświetlany na przycisku - "Zapisz".
    \end{itemize}
    \item Button (buttonback):
    \begin{itemize}
        \item android:id: Unikalny identyfikator dla przycisku "buttonback".
        \item android:layout\_width i android:layout\_height: Określają szerokość i wysokość przycisku na konkretne wartości.
        \item android:layout\_marginTop: Ustawia margines od góry na 500dp.
        \item android:backgroundTint: Ustawia kolor tła przycisku.
        \item android:fontFamily: Ustawia rodzinę czcionek dla tekstu na "sans-serif-light".
        \item android:text: Ustawia tekst na przycisku na wartość zasobu o nazwie "button14".
        \item android:textAppearance: Określa styl wyglądu tekstu na mały tekst.
    \end{itemize}
\end{itemize}
\subsection{Pliki values}
W folderze "values" w projekcie znajdują się zasoby związane z wartościami, takie jak kolory, ciągi znaków, style, itp. To miejsce, gdzie przechowuje się różne ustawienia i definicje, które są używane w aplikacji. 
\subsubsection{Plik colors.xml}
\hspace{0.60 cm}Zawiera definicje kolorów, które są używane w aplikacji. Można tu zdefiniować niestandardowe kolory i odwoływać się do nich w innych miejscach kodu.
\begin{lstlisting}[caption=colors.xml, label={lst:kod.xml}, language=XML]
	<resources>
	<color name="black">#FF000000</color>
	<color name="white">#FFFFFFFF</color>
	<color name="green">#0AA849</color>
	</resources>
\end{lstlisting}
\begin{itemize}
	\item Czarny (\#FF000000):
	\begin{itemize}
		\item name="black": Przypisuje nazwę "black" do tego koloru.
		\item \#FF000000: Reprezentuje kolor czarny w formie szesnastkowej. Składa się z czterech bajtów: jeden bajt na przezroczystość (FF - maksymalna nieprzezroczystość) i trzy bajty na wartości kanałów kolorów RGB (czerwony, zielony, niebieski).
	\end{itemize}
	\item Biały (\#FFFFFFFF):
	\begin{itemize}
		\item name="white": Przypisuje nazwę "white" do tego koloru.
		\item \#FFFFFFFF: Reprezentuje kolor biały. Podobnie jak w poprzednim przypadku, składa się z czterech bajtów: jeden bajt na przezroczystość (FF) i trzy bajty na wartości kanałów kolorów RGB.
	\end{itemize}
	\item Zielony (\#0AA849):
	\begin{itemize}
		\item name="green": Przypisuje nazwę "green" do tego koloru.
		\item \#0AA849: Reprezentuje niestandardowy odcień zielonego koloru. Składa się z czterech bajtów: jeden bajt na przezroczystość (0A), a trzy bajty na wartości kanałów kolorów RGB.
	\end{itemize}
\end{itemize}
\subsubsection{Plik strings.xml}
\hspace{0.60 cm} Przechowuje ciągi znaków używane w aplikacji. Zdefiniowane tu teksty mogą być używane w kodzie aplikacji, co ułatwia zarządzanie tekstem i tłumaczeniem.
\begin{lstlisting}[caption=strings.xml, label={lst:kod.xml}, language=XML]
	<resources>
	<string name="app_name">Finder</string>
	<string name="btn_scan">Skanuj</string>
	<string name="btn_info">Informacje</string>
	<string name="button2">Szukaj</string>
	<string name="button11">O nas</string>
	<string name="button12">Wersja</string>
	<string name="button13">Pomoc</string>
	<string name="button14">Powrot</string>
	<string name="start">Start</string>
	<string name="obraz1">tlo</string>
	<string name="obraz2">baner</string>
	</resources>
\end{lstlisting}
\begin{itemize}
	\item Nazwa aplikacji (app\_name):
	\begin{itemize}
		\item name="app\_name": Przypisuje nazwę "app\_name" do tej wartości.
		\item <string name="app\_name">Finder</string>: Określa nazwę aplikacji jako "Finder". Ta wartość jest często używana jako nazwa aplikacji wyświetlana w pasku tytułowym i innych miejscach.
	\end{itemize}
	\item Przycisk "Skanuj" (btn\_scan):
	\begin{itemize}
		\item name="btn\_scan": Przypisuje nazwę "btn\_scan" do tej wartości.
		\item <string name="btn\_scan">Skanuj</string>: Określa tekst na przycisku "Skanuj". Ta wartość może być używana wszędzie tam, gdzie potrzebny jest napis "Skanuj".
	\end{itemize}
	\item Przycisk "Informacje" (btn\_info):
	\begin{itemize}
		\item name="btn\_info": Przypisuje nazwę "btn\_info" do tej wartości.
		\item <string name="btn\_info">Informacje</string>: Określa tekst na przycisku "Informacje". Ta wartość może być używana wszędzie tam, gdzie potrzebny jest napis "Informacje".
	\end{itemize}
	\item Przycisk "Szukaj" (button2):
	\begin{itemize}
		\item name="button2": Przypisuje nazwę "button2" do tej wartości.
		\item <string name="button2">Szukaj</string>: Określa tekst na przycisku "Szukaj".
	\end{itemize}
	\item Inne przyciski o nazwach "button11", "button12", "button13", "button14":
	\begin{itemize}
		\item Przypisują nazwy i teksty do przycisków "O nas", "Wersja", "Pomoc" i "Powrót".
	\end{itemize}
	\item Inne wartości tekstowe o nazwach "start", "obraz1", "obraz2":
	\begin{itemize}
		\item Przypisują nazwy i teksty do innych elementów aplikacji, takich jak etykiety, opisy, czy nazwy obrazów.
	\end{itemize}
\end{itemize}
\subsubsection{Plik themes.xml}
Zawiera definicje motywów (themes), które określają wygląd ogólny aplikacji. Motywy są związane z kolorami, stylami i innymi ustawieniami interfejsu.
\begin{lstlisting}[caption=themes.xml, label={lst:kod.xml}, language=XML]
	<resources xmlns:tools="http://schemas.android.com/tools">
	<!--  Base application theme.  -->
	<style name="Base.Theme.Finder" parent="Theme.Material3.DayNight.NoActionBar">
	<!--  Customize your light theme here.  -->
	<!--  <item name="colorPrimary">@color/my_light_primary</item>  -->
	</style>
	<style name="Theme.Finder" parent="Base.Theme.Finder"/>
	</resources>
\end{lstlisting}
\begin{itemize}
	\item Styl bazowy aplikacji (Base.Theme.Finder):
	\begin{itemize}
		\item name="Base.Theme.Finder": Przypisuje nazwę "Base.Theme.Finder" do tego stylu.
		\item parent="Theme.Material3.DayNight.NoActionBar": Określa, że ten styl dziedziczy od wbudowanego stylu Theme.Material3.DayNight.NoActionBar. Styl ten jest bazowym motywem aplikacji, który obsługuje tryby dzień/noc (Day/Night) i nie zawiera paska akcji.
	\end{itemize}
	\item Dostosowany motyw aplikacji (Theme.Finder):
	\begin{itemize}
		\item name="Theme.Finder": Przypisuje nazwę "Theme.Finder" do tego stylu.
		\item parent="Base.Theme.Finder": Określa, że Theme.Finder dziedziczy od stylu bazowego Base.Theme.Finder. Jest to przykład dostosowanego motywu aplikacji, który może zawierać niestandardowe ustawienia kolorów, czcionek itp.
	\end{itemize}
	\item Zakomentowana sekcja (\(<!-- ... -->)\):
	\begin{itemize}
		\item Wewnętrzna sekcja stylu Base.Theme.Finder zawiera zakomentowany kod, który nie jest aktywny (\(<!-- ... -->)\). Jest to przykład, jak można dostosowywać konkretny kolor w motywie, ale w tym przypadku jest on wyłączony (zakomentowany przy użyciu <!-- ... -->). Można odkomentować i dostosować kolor, jeśli to potrzebne.
	\end{itemize}
\end{itemize}
\subsection{AndroidManifest.xml}
\hspace{0,60 cm} Plik AndroidManifest.xml to plik konfiguracyjny używany w projektach Android. Zawiera on zestaw informacji na temat aplikacji, jej komponentów i wymagań dotyczących środowiska wykonawczego. Plik AndroidManifest.xml jest kluczowy dla systemu Android, ponieważ dostarcza informacje, które są niezbędne do poprawnego funkcjonowania aplikacji, a także do komunikacji z innymi aplikacjami i systemem operacyjnym.
\begin{lstlisting}[caption=AndroidManifest.xml, label={lst:kod.xml}, language=XML]
<?xml version="1.0" encoding="utf-8"?>
<manifest xmlns:android="http://schemas.android.com/apk/res/android"
    xmlns:tools="http://schemas.android.com/tools">

    <!-- Uprawnienia -->
    <uses-permission android:name="android.permission.INTERNET" />
    <uses-permission android:name="android.permission.CAMERA" />
    <uses-feature android:name="android.hardware.camera" />
    <uses-permission android:name="android.permission.ACCESS_FINE_LOCATION" />
    <uses-permission android:name="android.permission.ACCESS_COARSE_LOCATION" />
    <uses-permission android:name="android.permission.POST_NOTIFICATIONS" />
    <uses-permission android:name="android.permission.VIBRATE" />
    <uses-feature android:name="android.hardware.camera.autofocus" />
    <uses-permission android:name="android.permission.NOTIFY" />
    <uses-permission android:name="android.permission.NOTIFICATION" />


    <application
        android:allowBackup="true"
        android:dataExtractionRules="@xml/data_extraction_rules"
        android:fullBackupContent="@xml/backup_rules"
        android:icon="@mipmap/ic_launcher"
        android:label="@string/app_name"
        android:roundIcon="@mipmap/ic_launcher_round"
        android:supportsRtl="true"
        android:theme="@style/Theme.Finder"
        tools:targetApi="31">

        <!-- Aktywnosc startowa -->
        <activity
            android:name=".StartActivity"
            android:exported="true"
            android:theme="@style/Theme.Finder">
            <intent-filter>
                <action android:name="android.intent.action.MAIN" />
                <category android:name="android.intent.category.LAUNCHER" />
            </intent-filter>
        </activity>
        
        <!-- Glowna aktywnosc -->
        <activity
            android:name=".MainActivity"
            android:exported="true"
            android:theme="@style/Theme.Finder">
        </activity>
    </application>
</manifest>
\end{lstlisting}

\begin{itemize}
    \item \(<?\)xml version="1.0" encoding="utf-8"\(?>\): Deklaracja wersji XML i kodowania pliku.
    \item \(<\)manifest xmlns:android="http://schemas.android.com/apk/res/android"\\xmlns:tools="http://schemas.android.com/tools"\(>\): Początek definicji pliku manifestu, gdzie xmlns:android i xmlns:tools to przestrzenie nazw używane w dokumencie.
    \item \(<!-- Uprawnienia -->\) : Komentarz informujący, że poniżej zaczynają się deklaracje uprawnień (permissions).
    \item \(<\)uses-permission android:name="android.permission.INTERNET" \(/>\): Uprawnienie dostępu do internetu.
    \item \(<\)uses-permission android:name="android.permission.INTERNET" \(/>\): Uprawnienie dostępu do internetu.
    \item \(<\)uses-permission android:name="android.permission.CAMERA" \(/>\): Uprawnienie dostępu do kamery.
    \item \(<\)uses-feature android:name="android.hardware.camera" \(/>\): Wymaganie funkcji kamery.
    \item \(<\)uses-permission android:name="android.permission.ACCESS\_FINE\_LOCATION" \(/>\): Uprawnienie dostępu do dokładnej lokalizacji.
    \item \(<\)uses-permission android:name="android.permission.ACCESS\_COARSE\_\\LOCATION" \(/>\): Uprawnienie dostępu do przybliżonej lokalizacji.
    \item \(<\)uses-permission android:name="android.permission.POST\_NOTIFICATIONS" \(/>\): Uprawnienie do publikowania powiadomień.
    \item \(<\)uses-permission android:name="android.permission.VIBRATE" \(/>\): Uprawnienie do wibracji.
    \item \(<\)uses-feature android:name="android.hardware.camera.autofocus" \(/>\): Wymaganie funkcji autofokusu kamery.
    \item \(<\)uses-permission android:name="android.permission.NOTIFY" \(/>\): Uprawnienie do powiadamiania.
    \item \(<\)uses-permission android:name="android.permission.NOTIFICATION" \(/>\): Uprawnienie do obsługi powiadomień.
    \item \(<\)application ...\(>\): Rozpoczęcie definicji aplikacji. Zawiera ustawienia globalne dla aplikacji.
    \item android:allowBackup="true": Włączenie automatycznego tworzenia kopii zapasowej danych aplikacji.
    \item android:dataExtractionRules="@xml/data\_extraction\_rules": Wskazanie reguł wydobywania danych aplikacji.
    \item android:fullBackupContent="@xml/backup\_rules": Wskazanie reguł pełnej kopii zapasowej danych.
    \item android:icon="@mipmap/ic\_launcher": Ustawienie ikony aplikacji.
    \item android:label="@string/app\_name": Ustawienie etykiety aplikacji.
    \item android:roundIcon="@mipmap/ic\_launcher\_round": Ustawienie okrągłej ikony aplikacji.
    \item android:supportsRtl="true": Wsparcie dla układów RTL (Right-to-Left).
    \item android:theme="@style/Theme.Finder": Ustawienie motywu aplikacji.
    \item tools:targetApi="31": Informacja narzędziowa o wersji API docelowej.
    \item \(<\)activity ...\(>\): Definicja aktywności (ekranów) w aplikacji.
    \item android:name=".StartActivity": Określenie nazwy klasy aktywności startowej.
    \item android:exported="true": Wskazuje, czy aktywność może być uruchamiana przez inne aplikacje.
    \item android:theme="@style/Theme.Finder": Ustawienie motywu dla tej konkretnej aktywności.
    \item \(<\)intent-filter\(>\): Określenie filtru intencji dla aktywności.
    \item \(<\)action android:name="android.intent.action.MAIN" \(/>\): Oznaczenie, że to jest główna aktywność.
    \item \(<\)category android:name="android.intent.category.LAUNCHER" \(/>\): Oznaczenie, że aplikacja ma ikonę na ekranie startowym.
    \item \(<\)activity android:name=".MainActivity" ...\(>\): Podobna definicja dla drugiej aktywności w aplikacji.
\end{itemize}