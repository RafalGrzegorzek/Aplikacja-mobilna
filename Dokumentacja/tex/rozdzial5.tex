	\newpage
\section{Testowanie}	%5
%Opisujemy testy, sprawdzamy czy nie generuje błędów.

	
	\begin{table}[ht]
		\centering
		\begin{tabular}{|c|p{5cm}|p{6cm}|p{2cm}|}
			\hline
			Lp. & Nazwa testu & Wynik testu & Rezultat \\
			\hline
			1. & Sprawdzanie powiadomień. & Powiadomienia pojawiają się na pasku powiadomień. & Zaliczony. \\ 
			\hline
			2. & Funkcjonalność przycisku "Skanuj QR" & Aplikacja otwiera aparat, po czym umożliwia zeskanowanie kodu QR. & Zaliczony. \\
			\hline
			3. & Przypisanie imienia użytkownika. & Wpisane imię zapisuje się w pamięci użądzenia. & Zaliczony. \\
			\hline
			4. & Powiadomienie z imieniem użytkownika. & Po przypisaniu imienia dla użytkownika, na pasku powiadomień aplikacja wita nas używając wpisanego imienia. & Zaliczony. \\
			\hline
			5. & Przyciski "Powrót" & Klikając przyciski "Powrót" aplikacja poprawnie cofa do poprzedniej podstrony. & Zaliczony. \\
			\hline
			6. & Funkcjonalność przycisku "Szukaj" & Po kliknięciu przycisku "Szukaj" aplikacja przekierowuje do Google Maps, i wyszukuje najbliższe sklepy "Żabka". & Zaliczony. \\
			\hline
			7. & Powiadomienie po zeskanowaniu QR. & Powiadomienie po zeskanowaniu kodu pojawia się na pasku powiadomień. & Zaliczony. \\
			\hline
			8. & Poprawne wyświetlanie się okienek w aplikacji. & Po naciśnięciu przycisków "O nas" lub "Wersja" wyskakuje okienko z informacjami, które można zamknąć naciskając "Ok". & Zaliczony. \\
			\hline
		\end{tabular}
		\caption{Tabela z wynikami testów}
		\label{tab:wyniki}
	\end{table}
	

